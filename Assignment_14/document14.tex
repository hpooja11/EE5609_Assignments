\documentclass[journal,12pt,twocolumn]{IEEEtran}

\usepackage{setspace}
\usepackage{gensymb}

\singlespacing
\usepackage{pstricks}
\usepackage[cmex10]{amsmath}
%\usepackage{amsthm}
%\interdisplaylinepenalty=2500
%\savesymbol{iint}
%\usepackage{txfonts}
%\restoresymbol{TXF}{iint}
%\usepackage{wasysym}
\usepackage{amsthm}
%\usepackage{iithtlc}
\usepackage{mathrsfs}
\usepackage{txfonts}
\usepackage{stfloats}
\usepackage{bm}
\usepackage{cite}
\usepackage{cases}
\usepackage{subfig}
%\usepackage{xtab}
\usepackage{longtable}
\usepackage{multirow}
%\usepackage{algorithm}
%\usepackage{algpseudocode}
\usepackage{enumitem}
\usepackage{mathtools}
\usepackage{caption}
\usepackage{steinmetz}
\usepackage{tikz}
%\usepackage{circuitikz}
\usepackage{verbatim}
\usepackage{tfrupee}
\usepackage[breaklinks=true]{hyperref}
%\usepackage{stmaryrd}
\usepackage{tkz-euclide} % loads  TikZ and tkz-base
%\usetkzobj{all}
\usetikzlibrary{calc,math}
\usepackage{listings}
\usepackage{color}                                            %%
\usepackage{array}                                            %%
\usepackage{longtable}                                        %%
\usepackage{calc}                                             %%
\usepackage{multirow}                                         %%
\usepackage{hhline}                                           %%
\usepackage{ifthen}                                           %%
%optionally (for landscape tables embedded in another document): %%
\usepackage{lscape}     
%\usepackage{multicol}
\usepackage{chngcntr}
%\usepackage{enumerate}
%\usepackage{wasysym}
%\newcounter{MYtempeqncnt}
\DeclareMathOperator*{\Res}{Res}
%\renewcommand{\baselinestretch}{2}
\renewcommand\thesection{\arabic{section}}
\renewcommand\thesubsection{\thesection.\arabic{subsection}}
\renewcommand\thesubsubsection{\thesubsection.\arabic{subsubsection}}

\renewcommand\thesectiondis{\arabic{section}}
\renewcommand\thesubsectiondis{\thesectiondis.\arabic{subsection}}
\renewcommand\thesubsubsectiondis{\thesubsectiondis.\arabic{subsubsection}}

% correct bad hyphenation here
\hyphenation{op-tical net-works semi-conduc-tor}
\def\inputGnumericTable{}                                 %%

\lstset{
	%language=C,
	frame=single, 
	breaklines=true,
	columns=fullflexible
}
\usepackage{amssymb}
\usepackage{stackengine}
\usepackage{scalerel}
\usepackage{graphicx}
\newlength\lthk
\setlength\lthk{.1ex}
\def\bline{\rule{2ex}{\lthk}}
\def\slash{\rotatebox{60}{\bline}}
\def\parallelogram{\stackMath\scalerel*{%
		\def\stackalignment{l}{\stackunder[-.5\lthk]{%
				\def\stackalignment{r}\stackon[-.5\lthk]{\slash\rule{.866ex}{0ex}\slash}{\bline}}%
			{\bline}}}{\square}%
}

\begin{document}
	%
	
	
	\newtheorem{theorem}{Theorem}[section]
	\newtheorem{problem}{Problem}
	\newtheorem{proposition}{Proposition}[section]
	\newtheorem{lemma}{Lemma}[section]
	\newtheorem{corollary}[theorem]{Corollary}
	\newtheorem{example}{Example}[section]
	\newtheorem{definition}[problem]{Definition}
	
	\newcommand{\BEQA}{\begin{eqnarray}}
		\newcommand{\EEQA}{\end{eqnarray}}
	\newcommand{\define}{\stackrel{\triangle}{=}}
	\bibliographystyle{IEEEtran}
	%\bibliographystyle{ieeetr}
	\providecommand{\mbf}{\mathbf}
	\providecommand{\pr}[1]{\ensuremath{\Pr\left(#1\right)}}
	\providecommand{\qfunc}[1]{\ensuremath{Q\left(#1\right)}}
	\providecommand{\sbrak}[1]{\ensuremath{{}\left[#1\right]}}
	\providecommand{\lsbrak}[1]{\ensuremath{{}\left[#1\right.}}
	\providecommand{\rsbrak}[1]{\ensuremath{{}\left.#1\right]}}
	\providecommand{\brak}[1]{\ensuremath{\left(#1\right)}}
	\providecommand{\lbrak}[1]{\ensuremath{\left(#1\right.}}
	\providecommand{\rbrak}[1]{\ensuremath{\left.#1\right)}}
	\providecommand{\cbrak}[1]{\ensuremath{\left\{#1\right\}}}
	\providecommand{\lcbrak}[1]{\ensuremath{\left\{#1\right.}}
	\providecommand{\rcbrak}[1]{\ensuremath{\left.#1\right\}}}
	\theoremstyle{remark}
	\newtheorem{rem}{Remark}
	\newcommand{\sgn}{\mathop{\mathrm{sgn}}}
	\providecommand{\abs}[1]{\left\vert#1\right\vert}
	\providecommand{\res}[1]{\Res\displaylimits_{#1}} 
	\providecommand{\norm}[1]{\left\lVert#1\right\rVert}
	%\providecommand{\norm}[1]{\lVert#1\rVert}
	\providecommand{\mtx}[1]{\mathbf{#1}}
	\providecommand{\mean}[1]{E\left[ #1 \right]}
	\providecommand{\fourier}{\overset{\mathcal{F}}{ \rightleftharpoons}}
	%\providecommand{\hilbert}{\overset{\mathcal{H}}{ \rightleftharpoons}}
	\providecommand{\system}{\overset{\mathcal{H}}{ \longleftrightarrow}}
	%\newcommand{\solution}[2]{\textbf{Solution:}{#1}}
	\newcommand{\solution}{\noindent \textbf{Solution: }}
	\newcommand{\cosec}{\,\text{cosec}\,}
	\providecommand{\dec}[2]{\ensuremath{\overset{#1}{\underset{#2}{\gtrless}}}}
	\newcommand{\myvec}[1]{\ensuremath{\begin{pmatrix}#1\end{pmatrix}}}
	\newcommand{\mydet}[1]{\ensuremath{\begin{vmatrix}#1\end{vmatrix}}}
	%\numberwithin{equation}{section}
	\numberwithin{equation}{subsection}
	%\numberwithin{problem}{section}
	%\numberwithin{definition}{section}
	\makeatletter
	\@addtoreset{figure}{problem}
	\makeatother
	\let\StandardTheFigure\thefigure
	\let\vec\mathbf
	%\renewcommand{\thefigure}{\theproblem.\arabic{figure}}
	\renewcommand{\thefigure}{\theproblem}
	%\setlist[enumerate,1]{before=\renewcommand\theequation{\theenumi.\arabic{equation}}
	%\counterwithin{equation}{enumi}
	%\renewcommand{\theequation}{\arabic{subsection}.\arabic{equation}}
	\def\putbox#1#2#3{\makebox[0in][l]{\makebox[#1][l]{}\raisebox{\baselineskip}[0in][0in]{\raisebox{#2}[0in][0in]{#3}}}}
	\def\rightbox#1{\makebox[0in][r]{#1}}
	\def\centbox#1{\makebox[0in]{#1}}
	\def\topbox#1{\raisebox{-\baselineskip}[0in][0in]{#1}}
	\def\midbox#1{\raisebox{-0.5\baselineskip}[0in][0in]{#1}}
	\vspace{3cm}
	\title{Assignment-14}
	\author{Pooja H \\ }
	\maketitle
	\newpage
	\bigskip
	\renewcommand{\thefigure}{\theenumi}
	\renewcommand{\thetable}{\theenumi}
	\begin{abstract}
		In this document, we find the matrix of $T$ in the ordered basis $ \vec{B}$
	\end{abstract}
	%Download all python codes from 
	%\begin{lstlisting}
	%	https://github.com/poojah15/EE5609_Assignments/tree/master/Assignment_13
	%\end{lstlisting}
	Download all latex-tikz codes from 
\begin{lstlisting}
	https://github.com/poojah15/EE5609_Assignments/tree/master/Assignment_14
\end{lstlisting}
	\section{Problem Statement}
	Let $T$ be the linear operator on $\vec{R}^2$ defined by 
	\begin{align}
		T\myvec{x_1\\ x_2} = \myvec{-x_2\\ x_1} 
	\end{align}
	What is the matrix of $T$ in the ordered basis $\vec{B} = \cbrak{ \vec{ \alpha }_1, \vec{\alpha}_2}$, where $\vec{\alpha}_1 = \myvec{1\\ 2}$ and $\vec{\alpha}_2 = \myvec{1 \\ -1}$ ?
	\section{Solution}
	Let $\vec{B}'$ be the standard ordered basis for $\vec{R}^2$. Then,
	\begin{align}
		T(\vec{\alpha}'_1) &= T\myvec{1 \\ 0} = \myvec{0\\1} = 0\vec{\alpha}_1 + 1\vec{\alpha}_2\\
		T(\vec{\alpha}'_2) &= T\myvec{0 \\ 1} = \myvec{-1\\0} = -1\vec{\alpha}_1 + 0\vec{\alpha}_2
	\end{align}
Hence, the matrix of $T$ in the standard ordered basis $\vec{B}'$ is
\begin{align}
	T_\vec{B'} = \myvec{0 & 1\\ -1 & 0}
\end{align}
Given, 
\begin{align}
	\vec{\alpha}_1 &= \myvec{1\\ 2} \\ 
	\vec{\alpha}_2 &= \myvec{1 \\ -1} 
\end{align}
Then,
\begin{align}
	\vec{\alpha}'_1 &= 1\vec{\alpha}_1 + 2\vec{ \alpha }_2\\
	\vec{\alpha}'_2 &= 1\vec{\alpha}_1 - 1\vec{ \alpha }_2
\end{align}		
So the $\vec{P}$ matrix is
\begin{align}
	\vec{P} = \myvec{1 & 1\\ 2 & -1}
\end{align}	
And
\begin{align}
	\vec{P}^{-1} = \myvec{\frac{1}{3} & \frac{1}{3} \\ \frac{2}{3} & -\frac{1}{3}}
\end{align}
Hence
\begin{align}
	[T]_\vec{B} &= \vec{P}^{-1} [T]_\vec{B'} \vec{P}\\
	&= \myvec{\frac{1}{3} & \frac{1}{3} \\ \frac{2}{3} & -\frac{1}{3}}\myvec{0 & 1\\ -1 & 0}\myvec{1 & 1\\ 2 & -1}\\
	&= \myvec{-\frac{1}{3} & \frac{1}{3} \\ \frac{1}{3} & \frac{2}{3}}\myvec{1 & 1\\ 2 & -1}\\
	[T]_\vec{B} &= \myvec{\frac{1}{3} & -\frac{2}{3} \\ \frac{5}{3} & -\frac{1}{3}}
\end{align}	
Hence, $[T]_\vec{B}$ is the required matrix for the given ordered basis $\vec{B}$.
\end{document}
