\documentclass[journal,12pt,twocolumn]{IEEEtran}

\usepackage{setspace}
\usepackage{gensymb}

\singlespacing


\usepackage[cmex10]{amsmath}
%\usepackage{amsthm}
%\interdisplaylinepenalty=2500
%\savesymbol{iint}
%\usepackage{txfonts}
%\restoresymbol{TXF}{iint}
%\usepackage{wasysym}
\usepackage{amsthm}
%\usepackage{iithtlc}
\usepackage{mathrsfs}
\usepackage{txfonts}
\usepackage{stfloats}
\usepackage{bm}
\usepackage{cite}
\usepackage{cases}
\usepackage{subfig}
%\usepackage{xtab}
\usepackage{longtable}
\usepackage{multirow}
%\usepackage{algorithm}
%\usepackage{algpseudocode}
\usepackage{enumitem}
\usepackage{mathtools}
\usepackage{steinmetz}
\usepackage{tikz}
%\usepackage{circuitikz}
\usepackage{verbatim}
\usepackage{tfrupee}
\usepackage[breaklinks=true]{hyperref}
%\usepackage{stmaryrd}
\usepackage{tkz-euclide} % loads  TikZ and tkz-base
%\usetkzobj{all}
\usetikzlibrary{calc,math}
\usepackage{listings}
   \usepackage{color}                                            %%
    \usepackage{array}                                            %%
    \usepackage{longtable}                                        %%
    \usepackage{calc}                                             %%
    \usepackage{multirow}                                         %%
    \usepackage{hhline}                                           %%
    \usepackage{ifthen}                                           %%
  %optionally (for landscape tables embedded in another document): %%
    \usepackage{lscape}     
%\usepackage{multicol}
\usepackage{chngcntr}
%\usepackage{enumerate}

%\usepackage{wasysym}
%\newcounter{MYtempeqncnt}
\DeclareMathOperator*{\Res}{Res}
%\renewcommand{\baselinestretch}{2}
\renewcommand\thesection{\arabic{section}}
\renewcommand\thesubsection{\thesection.\arabic{subsection}}
\renewcommand\thesubsubsection{\thesubsection.\arabic{subsubsection}}

\renewcommand\thesectiondis{\arabic{section}}
\renewcommand\thesubsectiondis{\thesectiondis.\arabic{subsection}}
\renewcommand\thesubsubsectiondis{\thesubsectiondis.\arabic{subsubsection}}

% correct bad hyphenation here
\hyphenation{op-tical net-works semi-conduc-tor}
\def\inputGnumericTable{}                                 %%

\lstset{
%language=C,
frame=single, 
breaklines=true,
columns=fullflexible
}

\begin{document}
%


\newtheorem{theorem}{Theorem}[section]
\newtheorem{problem}{Problem}
\newtheorem{proposition}{Proposition}[section]
\newtheorem{lemma}{Lemma}[section]
\newtheorem{corollary}[theorem]{Corollary}
\newtheorem{example}{Example}[section]
\newtheorem{definition}[problem]{Definition}

\newcommand{\BEQA}{\begin{eqnarray}}
\newcommand{\EEQA}{\end{eqnarray}}
\newcommand{\define}{\stackrel{\triangle}{=}}
\bibliographystyle{IEEEtran}
%\bibliographystyle{ieeetr}
\providecommand{\mbf}{\mathbf}
\providecommand{\pr}[1]{\ensuremath{\Pr\left(#1\right)}}
\providecommand{\qfunc}[1]{\ensuremath{Q\left(#1\right)}}
\providecommand{\sbrak}[1]{\ensuremath{{}\left[#1\right]}}
\providecommand{\lsbrak}[1]{\ensuremath{{}\left[#1\right.}}
\providecommand{\rsbrak}[1]{\ensuremath{{}\left.#1\right]}}
\providecommand{\brak}[1]{\ensuremath{\left(#1\right)}}
\providecommand{\lbrak}[1]{\ensuremath{\left(#1\right.}}
\providecommand{\rbrak}[1]{\ensuremath{\left.#1\right)}}
\providecommand{\cbrak}[1]{\ensuremath{\left\{#1\right\}}}
\providecommand{\lcbrak}[1]{\ensuremath{\left\{#1\right.}}
\providecommand{\rcbrak}[1]{\ensuremath{\left.#1\right\}}}
\theoremstyle{remark}
\newtheorem{rem}{Remark}
\newcommand{\sgn}{\mathop{\mathrm{sgn}}}
\providecommand{\abs}[1]{\left\vert#1\right\vert}
\providecommand{\res}[1]{\Res\displaylimits_{#1}} 
\providecommand{\norm}[1]{\left\lVert#1\right\rVert}
%\providecommand{\norm}[1]{\lVert#1\rVert}
\providecommand{\mtx}[1]{\mathbf{#1}}
\providecommand{\mean}[1]{E\left[ #1 \right]}
\providecommand{\fourier}{\overset{\mathcal{F}}{ \rightleftharpoons}}
%\providecommand{\hilbert}{\overset{\mathcal{H}}{ \rightleftharpoons}}
\providecommand{\system}{\overset{\mathcal{H}}{ \longleftrightarrow}}
	%\newcommand{\solution}[2]{\textbf{Solution:}{#1}}
\newcommand{\solution}{\noindent \textbf{Solution: }}
\newcommand{\cosec}{\,\text{cosec}\,}
\providecommand{\dec}[2]{\ensuremath{\overset{#1}{\underset{#2}{\gtrless}}}}
\newcommand{\myvec}[1]{\ensuremath{\begin{pmatrix}#1\end{pmatrix}}}
\newcommand{\mydet}[1]{\ensuremath{\begin{vmatrix}#1\end{vmatrix}}}
%\numberwithin{equation}{section}
\numberwithin{equation}{subsection}
%\numberwithin{problem}{section}
%\numberwithin{definition}{section}
\makeatletter
\@addtoreset{figure}{problem}
\makeatother
\let\StandardTheFigure\thefigure
\let\vec\mathbf
%\renewcommand{\thefigure}{\theproblem.\arabic{figure}}
\renewcommand{\thefigure}{\theproblem}
%\setlist[enumerate,1]{before=\renewcommand\theequation{\theenumi.\arabic{equation}}
%\counterwithin{equation}{enumi}
%\renewcommand{\theequation}{\arabic{subsection}.\arabic{equation}}
\def\putbox#1#2#3{\makebox[0in][l]{\makebox[#1][l]{}\raisebox{\baselineskip}[0in][0in]{\raisebox{#2}[0in][0in]{#3}}}}
     \def\rightbox#1{\makebox[0in][r]{#1}}
     \def\centbox#1{\makebox[0in]{#1}}
     \def\topbox#1{\raisebox{-\baselineskip}[0in][0in]{#1}}
     \def\midbox#1{\raisebox{-0.5\baselineskip}[0in][0in]{#1}}
\vspace{3cm}
\title{Assignment-11}
\author{Pooja H \\ }
\maketitle
\newpage
\bigskip
\renewcommand{\thefigure}{\theenumi}
\renewcommand{\thetable}{\theenumi}
\newenvironment{amatrix}[1]{%
	\left(\begin{array}{@{}*{#1}{c}|c@{}}
	}{%
	\end{array}\right)
}

\begin{abstract}
In this document, we find the basis for the space V
\end{abstract}
Download all latex-tikz codes from 
\begin{lstlisting}
https://github.com/poojah15/EE5609_Assignments/tree/master/Assignment_11
\end{lstlisting}


\section{Problem Statement}
Let $\vec{V}$ be the space of $2\times2$ matrices over $\vec{F}$. Find a basis $\cbrak{\vec{A}_1, \vec{A}_2, \vec{A}_3, \vec{A}_4}$ for $\vec{V}$ such that $\vec{A}_j^2 = \vec{A}_j$ for each $j$
\section{Solution}
Every 2$\times$2 matrix may be written as
\begin{align}
	\myvec{a & b\\c & d} = a\myvec{1 & 0\\0 & 0} + b\myvec{0 & 1\\0 & 0} + c\myvec{0& 0\\1 & 0} + d\myvec{0 & 0\\0 & 1}
\end{align}
This shows that 
\begin{align}
	\cbrak{\vec{A}_1, \vec{A}_2, \vec{A}_3, \vec{A}_4} = \cbrak{\myvec{1 & 0\\0 & 0}, \myvec{0 & 1\\0 & 0}, \myvec{0& 0\\1 & 0}, \myvec{0 & 0\\0 & 1}}
\end{align}
can be the basis for the space $\vec{V}$ of all $2\times2$ matrices. However $\vec{A}_2$ and $\vec{A}_3$ doesn't satisfy the property of $\vec{A}^2 = \vec{A}$. Consider b = 0 and c = 0, then the matrix 
\begin{align}
	\myvec{a & 0\\ 0& d} 
\end{align}
can't be a basis as it is the linear combination of $\vec{A}_1$ and $\vec{A}_4$. Hence either b or c or both must be non zero. Hence,
\begin{align}
	\vec{A}_2 &= \myvec{1 & 0\\1 & 0}\\
	\vec{A}_3 &=  \myvec{0 & 1\\0 & 1}
\end{align}
Here, $\vec{A}_2^2 = \vec{A}_2$ and $\vec{A}_3^2 = \vec{A}_3$. Therefore the basis can be
\begin{align}
	\cbrak{\vec{A}_1, \vec{A}_2, \vec{A}_3, \vec{A}_4} = \cbrak{\myvec{1 & 0\\0 & 0}, \myvec{1 & 0\\1 & 0}, \myvec{0& 1\\0 & 1}, \myvec{0 & 0\\0 & 1}}
\end{align}
$\cbrak{\vec{A}_1, \vec{A}_2, \vec{A}_3, \vec{A}_4}$ forms the basis, iff they are linearly independent and the linear combination of them span the space $\vec{V}$. To show that they are linearly independent, we show that the equation has a trivial solution.
\begin{align}
	a\myvec{1 & 0\\0 & 0} + b\myvec{1 & 0\\1 & 0} + c\myvec{0& 1\\0 & 1} + d\myvec{0 & 0\\0 & 1} &= \myvec{0 & 0 \\ 0 & 0}\\
	\implies a + b &= 0\\ 
	b &= 0\\
	c &= 0\\ 
	c + d &= 0
\end{align}
The corresponding matrix form is $\vec{Ax} = 0$
\begin{align}
	\myvec{1 & 1& 0 & 0\\ 0 & 0 & 1 & 0\\ 0 & 1 & 0 & 0\\0 & 0 & 1 & 1}\myvec{a\\b\\c\\d} = \myvec{0\\0\\0\\0}
\end{align}
Row reducing the augmented matrix,
\begin{align}
	\myvec{1 & 1& 0 & 0& 0\\ 0 & 0 & 1 & 0& 0\\ 0 & 1 & 0 & 0 &0\\0 & 0 & 1 & 1 & 0}
	\xleftrightarrow[R_4 \leftarrow R_4 - R_3]{R_2 \xleftrightarrow{} R_3}
	\myvec{1 & 1& 0 & 0& 0\\ 0 & 1 & 0 & 0& 0\\ 0 & 0 & 1 & 0 &0\\0 & 0 & 0 & 1 & 0}\\
	\xleftrightarrow{R_1 \leftarrow R_1 - R_2}
	\myvec{1 & 0& 0 & 0& 0\\ 0 & 1 & 0 & 0& 0\\ 0 & 0 & 1 & 0 &0\\0 & 0 & 0 & 1 & 0}
\end{align}
Therefore, a = b = c = d = 0. Hence the matrices are linearly independent. To show that the linear combination of $\cbrak{\vec{A}_1, \vec{A}_2, \vec{A}_3, \vec{A}_4} $ span the space $\vec{V}$, consider an arbitrary matrix,
\begin{align}
	\myvec{w & x\\ y & z}
\end{align} 
Compute a, b, c, d such that
\begin{align}
	\myvec{w & x\\ y & z} &= a\myvec{1 & 0\\0 & 0} + b\myvec{1 & 0\\1 & 0} + c\myvec{0& 1\\0 & 1} + d\myvec{0 & 0\\0 & 1} \label{eq:eq1}\\
	&= \myvec{a + b & c \\ b & c + d}
\end{align}
Equating the entries, this produces system of linear equations,
\begin{align}
	a + b &= w, y = b, x = c, z = c + d\\
	\implies a &= w - y\\
	 b &= y\\
	  c &= x\\
	  d &= z - x
\end{align}
In particular, there exists atleast one solution regardless of the values of w, x, y, z.
For example, consider the following matrix,
\begin{align}
	\myvec{w & x\\ y & z} = \myvec{3 & 4 \\ -2 & 7}
\end{align}
Here, $a = 5, b = -2, c = 4, d = 3$. Using \eqref{eq:eq1}, we get
\begin{align}
	5\myvec{1& 0\\0 & 0} - 2\myvec{1 & 0\\1 & 0} + 4\myvec{0& 1\\0 & 1} + 3\myvec{0 & 0\\0 & 1} = \myvec{3 & 4 \\ -2 & 7}
\end{align}
Hence $\cbrak{\myvec{1 & 0\\0 & 0}, \myvec{1 & 0\\1 & 0}, \myvec{0& 1\\0 & 1}, \myvec{0 & 0\\0 & 1}}$ forms the basis for the given space $\vec{V}$.
\end{document}
